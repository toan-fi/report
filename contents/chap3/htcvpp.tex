Để đảm bảo dự án được triển khai có hệ thống, đúng tiến độ và đáp ứng tốt yêu cầu thực tế của người dùng, nhóm áp dụng phương pháp phát triển phần mềm theo hướng \textbf{Agile}, kết hợp với các công cụ cộng tác trực tuyến nhằm tối ưu hiệu quả làm việc nhóm và quản lý dự án.

\subsection{Quy trình phát triển theo hướng Agile}
Dự án được triển khai theo mô hình \textbf{làm việc theo milestone}, phù hợp với quy mô nhóm gồm 4 thành viên và thời gian thực hiện giới hạn. Thay vì phát triển toàn bộ hệ thống trong một giai đoạn duy nhất, nhóm chia dự án thành nhiều milestone nhỏ, mỗi milestone tập trung giải quyết một tập chức năng hoặc mục tiêu cụ thể (ví dụ: phân tích yêu cầu, xây dựng MVP, hoàn thiện và tối ưu).

Cách tiếp cận này giúp nhóm:
\begin{itemize}
    \item Dễ dàng theo dõi tiến độ phát triển của dự án.
    \item Linh hoạt điều chỉnh phạm vi và ưu tiên tính năng khi có thay đổi về yêu cầu.
    \item Phát hiện sớm các vấn đề kỹ thuật hoặc thiết kế để kịp thời cải tiến.
\end{itemize}
Sau mỗi milestone, nhóm tiến hành rà soát lại kết quả đạt được, đánh giá mức độ hoàn thành so với mục tiêu ban đầu và thống nhất kế hoạch cho giai đoạn tiếp theo.

\subsection{Công cụ cộng tác và quản lý dự án}
Nhóm sử dụng \textbf{GitHub} làm nền tảng trung tâm cho việc quản lý mã nguồn và phối hợp phát triển. Cụ thể:
\begin{itemize}
    \item \textbf{GitHub Repository} được dùng để lưu trữ toàn bộ mã nguồn của dự án, đảm bảo tính nhất quán và dễ dàng kiểm soát phiên bản.
    \item \textbf{GitHub Issues} được sử dụng để ghi nhận các công việc cần thực hiện, lỗi phát sinh và đề xuất cải tiến. Mỗi issue được mô tả rõ ràng và gán cho thành viên phụ trách.
    \item \textbf{Pull Request (PR)} được áp dụng trong quá trình phát triển nhằm đảm bảo mã nguồn được rà soát trước khi hợp nhất vào nhánh chính. Điều này giúp giảm lỗi, nâng cao chất lượng code và tăng tính minh bạch trong làm việc nhóm.
    \item \textbf{GitHub Projects} được sử dụng như một bảng quản lý công việc (Kanban), trong đó các issue được sắp xếp theo từng milestone và trạng thái (To Do – In Progress – Done), giúp cả nhóm theo dõi tiến độ một cách trực quan.
\end{itemize}

Bên cạnh đó, nhóm kết hợp các công cụ giao tiếp để hỗ trợ trao đổi nhanh và họp nhóm:
\begin{itemize}
    \item \textbf{Zalo} được sử dụng cho trao đổi hằng ngày, thảo luận nhanh các vấn đề phát sinh.
    \item \textbf{Google Meet} được dùng cho các buổi họp nhóm định kỳ, thống nhất hướng phát triển, phân công công việc và đánh giá kết quả sau mỗi milestone.
\end{itemize}

\subsection{Cơ sở ra quyết định và phân công công việc}
Các quyết định quan trọng trong quá trình phát triển (phạm vi tính năng, thứ tự ưu tiên, điều chỉnh kế hoạch) đều được thảo luận và thống nhất trong nhóm, dựa trên:
\begin{itemize}
    \item Kết quả nghiên cứu thị trường và nhu cầu thực tế của sinh viên.
    \item Mức độ tác động của tính năng đến người dùng (Impact) và khả năng triển khai trong thời gian cho phép.
    \item Mục tiêu xây dựng một phiên bản \textbf{MVP ổn định, dễ mở rộng} trong tương lai.
\end{itemize}
Việc phân công công việc được thực hiện dựa trên milestone và issue, đảm bảo mỗi thành viên đều có trách nhiệm rõ ràng nhưng vẫn hỗ trợ lẫn nhau khi cần thiết. Cách làm này giúp nhóm duy trì tiến độ ổn định, đồng thời tăng tính phối hợp và trách nhiệm cá nhân trong suốt quá trình phát triển dự án.
