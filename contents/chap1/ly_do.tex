\subsection{Lý do chọn dự án}
Trong quá trình học tập tại Trường Đại học Bách Khoa, sinh viên thường xuyên có nhu cầu trao đổi tài liệu học tập và dụng cụ học tập như giáo trình, đề thi, sách tham khảo, linh kiện và các thiết bị học tập cũ. Hiện nay, các hoạt động này chủ yếu được thực hiện thông qua các nền tảng mạng xã hội phổ biến như Facebook, Zalo hoặc Telegram.

Tuy nhiên, việc sử dụng các nền tảng không chuyên biệt này bộc lộ nhiều hạn chế. Các thông tin được đăng tải rời rạc, thiếu tổ chức và khó tìm kiếm lại khi cần thiết. Bài viết nhanh chóng bị trôi theo dòng thời gian, dẫn đến việc người dùng dễ bỏ lỡ những tài nguyên học tập quan trọng. Bên cạnh đó, các nền tảng này không được thiết kế riêng cho mục đích chia sẻ học liệu, nên chưa đáp ứng tốt nhu cầu thực tế của sinh viên trong môi trường học tập đại học. Hiện tại, vẫn chưa tồn tại một nền tảng nội bộ chính thức dành riêng cho sinh viên trong trường để phục vụ việc chia sẻ và trao đổi tài nguyên học tập.

Xuất phát từ nhu cầu thực tiễn đó, StudyShare được xây dựng nhằm cung cấp một giải pháp tập trung, rõ ràng và tiện lợi cho sinh viên Trường Đại học Bách Khoa. Ứng dụng đóng vai trò như một không gian chia sẻ học tập chung, nơi sinh viên có thể dễ dàng đăng tải, tìm kiếm và trao đổi tài liệu cũng như dụng cụ học tập thông qua các hình thức như cho mượn, mua bán hoặc tặng lại. Việc tập trung hóa các hoạt động này giúp tiết kiệm thời gian tìm kiếm, nâng cao hiệu quả học tập và giảm sự phụ thuộc vào các nền tảng mạng xã hội không chuyên biệt.

Nhu cầu chia sẻ tài nguyên học tập là nhu cầu thiết yếu và diễn ra thường xuyên, đặc biệt đối với sinh viên các ngành kỹ thuật và công nghệ. Thông qua việc chuẩn hóa quy trình trao đổi và tổ chức thông tin một cách khoa học, StudyShare không chỉ giúp sinh viên tiếp cận tài nguyên học tập nhanh chóng hơn mà còn khuyến khích tinh thần chia sẻ và tái sử dụng trong cộng đồng sinh viên. Điều này góp phần giảm chi phí học tập, hạn chế lãng phí và thúc đẩy sự gắn kết giữa các sinh viên trong cùng khoa và toàn trường.

Về lâu dài, StudyShare có tiềm năng phát triển thành một nền tảng chia sẻ học liệu uy tín trong nội bộ sinh viên Khoa học Máy tính, từ đó mở rộng mô hình cho toàn bộ sinh viên Trường Đại học Bách Khoa, hướng tới xây dựng một hệ sinh thái học tập bền vững và hiệu quả.
