\subsection{Business Model Canvas}
\subsubsection{Ý tưởng (IDEA)}

\begin{table}[H]
\centering
\renewcommand{\arraystretch}{1.3}
\begin{tabular}{|p{5cm}|p{5cm}|p{5cm}|}
\hline
\textbf{Vấn đề (Problem)} &
\textbf{Giải pháp (Solution)} &
\textbf{Giá trị mang lại (Value)} \\
\hline
-- Sinh viên Trường Đại học Bách Khoa có nhu cầu trao đổi, mua bán hoặc chia sẻ tài liệu và dụng cụ học tập đã qua sử dụng, nhưng chưa có nền tảng chuyên biệt và an toàn.

\par\smallskip
-- Các nền tảng như Facebook Group hoặc Facebook Marketplace không được thiết kế riêng cho môi trường sinh viên, dẫn đến thông tin pha tạp, khó tìm kiếm và thiếu kiểm soát.

\par\smallskip
-- Nhiều sinh viên lãng phí tài nguyên học tập do không có kênh chuyển giao phù hợp.
&
-- Mô hình chia sẻ học liệu nội bộ \textbf{StudyShare}, được xây dựng dành riêng cho sinh viên Trường Đại học Bách Khoa.

\par\smallskip
-- Nền tảng cho phép đăng tải, tìm kiếm và trao đổi học liệu trong phạm vi nội bộ, có xác thực tài khoản sinh viên.
&
-- Giảm thời gian tìm kiếm và tiếp cận học liệu.

\par\smallskip
-- Tăng tỷ lệ tái sử dụng tài nguyên học tập trong cộng đồng sinh viên.

\par\smallskip
-- Giảm chi phí học tập trung bình của sinh viên.

\par\smallskip
-- Nâng cao tính an toàn và minh bạch trong trao đổi học liệu.
\\
\hline
\end{tabular}
\end{table}





\subsubsection{Bán hàng (SELL)}

\begin{table}[H]
\centering
\renewcommand{\arraystretch}{1.3}
\begin{tabular}{|p{6cm}|p{4.5cm}|p{5cm}|}
\hline
\textbf{Lợi thế cạnh tranh (Advantage)} &
\textbf{Đối tượng người dùng (Customer/User)} &
\textbf{Chỉ số đo lường thành công (Metrics)} \\
\hline
-- Cộng đồng nội bộ được xác thực bằng email sinh viên \texttt{@hcmut.edu.vn}.

\par\smallskip
-- Bộ lọc đặc thù cho trao đổi giữa sinh viên, cho phép tra cứu và lọc theo môn học và ngành học.

\par\smallskip
-- Môi trường đáng tin cậy có kiểm duyệt nhằm hạn chế bài đăng spam hoặc người ngoài trường.

\par\smallskip
-- Tích hợp tính năng đánh giá và bình luận để tăng độ tin cậy trong quá trình trao đổi.

\par\smallskip
-- Cho phép người dùng nhắn tin trực tiếp trong ứng dụng.
&
-- Sinh viên Đại học Bách Khoa TP.HCM từ năm nhất đến năm cuối và sinh viên sắp ra trường.
&
-- Số lượng học liệu hoặc dụng cụ được đăng tải mỗi tháng.

\par\smallskip
-- Số lượng học liệu hoặc tài liệu trao đổi thành công mỗi tháng.

\par\smallskip
-- Mức độ hài lòng của người dùng sau khi sử dụng.

\par\smallskip
-- Tỷ lệ tái sử dụng học liệu qua các học kỳ.

\par\smallskip
-- Tỷ lệ người dùng quay lại (Retention Rate) hàng tháng.
\\
\hline
\end{tabular}
\end{table}





\subsubsection{Mô hình kinh doanh (BUSINESS MODEL)}

\begin{table}[H]
\centering
\renewcommand{\arraystretch}{1.3}
\begin{tabular}{|p{5cm}|p{5cm}|p{5cm}|}
\hline
\textbf{Kênh tiếp cận (Channels)} &
\textbf{Chi phí vận hành (Costs)} &
\textbf{Doanh thu (Revenue)} \\
\hline
-- Truyền thông qua các \textbf{Câu lạc bộ sinh viên, fanpage Khoa, group Facebook chính thức của trường}.

\par\smallskip
-- Marketing nội bộ trong khuôn viên trường (poster, sự kiện, truyền miệng).

\par\smallskip
-- Tận dụng email sinh viên để gửi mail mời đăng ký tài khoản.

\par\smallskip
-- Xây dựng các kênh mạng xã hội để quảng bá (Tiktok, Facebook).
&
-- Chi phí hạ tầng hệ thống (hosting, lưu trữ dữ liệu, bảo mật).

\par\smallskip
-- Chi phí quản lý cộng đồng và kiểm duyệt nội dung.

\par\smallskip
-- Chi phí truyền thông nội bộ và hỗ trợ kỹ thuật.
&
-- Giai đoạn đầu (MVP): miễn phí hoàn toàn.

\par\smallskip
-- Giai đoạn sau có thể mở rộng:

\par\smallskip
\quad -- Quảng cáo nội bộ (ưu tiên bài đăng).

\par\smallskip
\quad -- Gói ``đẩy tin'' hoặc ``nổi bật'' cho người bán.

\par\smallskip
\quad -- Phí dịch vụ nhỏ cho các giao dịch có hỗ trợ trung gian.
\\
\hline
\end{tabular}
\end{table}





\subsubsection{Giá trị nổi bật (Unique Value Proposition)}

\begin{table}[H]
\centering
\renewcommand{\arraystretch}{1.3}
\begin{tabular}{|p{6cm}|p{8cm}|}
\hline
\textbf{StudyShare} &
\textbf{Khác biệt so với nền tảng hiện tại} \\
\hline
Nội bộ sinh viên Bách Khoa (xác thực email \texttt{@hcmut.edu.vn}) &
So với Facebook, Chợ Tốt, Oreka -- nền tảng công khai, thiếu kiểm duyệt. \\
\hline
Tối ưu cho giao dịch học liệu và dụng cụ học tập &
Không bị nhiễu bởi các bài đăng ngoài mục đích học tập. \\
\hline
Chat trực tiếp trong ứng dụng &
Không cần qua nền tảng thứ ba, tiết kiệm thời gian. \\
\hline
Tập trung vào trải nghiệm di động đơn giản, thân thiện &
Phù hợp với thói quen sử dụng điện thoại của sinh viên. \\
\hline
Phát triển theo hướng cộng đồng học tập &
Hướng đến xây dựng văn hóa chia sẻ và hỗ trợ lẫn nhau giữa các khóa. \\
\hline
\end{tabular}
\end{table}
