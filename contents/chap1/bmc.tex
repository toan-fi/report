\subsection{Business Model Canvas}
\subsubsection{Ý tưởng (IDEA)}

\begin{table}[H]
\centering
\renewcommand{\arraystretch}{1.3}
\begin{tabular}{|p{5cm}|p{5cm}|p{5cm}|}
\hline
\textbf{Vấn đề (Problem)} &
\textbf{Giải pháp (Solution)} &
\textbf{Giá trị mang lại (Value)} \\
\hline
\begin{itemize}
    \item Sinh viên Trường Đại học Bách Khoa có nhu cầu trao đổi, mua bán hoặc chia sẻ lại tài liệu và dụng cụ học tập đã qua sử dụng, nhưng hiện chưa có một nền tảng chuyên biệt và đảm bảo an toàn cho mục đích này.
    \item Các nền tảng phổ biến như Facebook Group hoặc Facebook Marketplace không được thiết kế riêng cho môi trường sinh viên, dẫn đến việc thông tin bị pha tạp, khó tìm kiếm và thiếu kiểm soát.
    \item Nhiều sinh viên lãng phí tài nguyên học tập (sách, dụng cụ, áo blouse) do không có kênh phù hợp để chuyển giao cho các khóa sau.
\end{itemize}
&
\begin{itemize}
    \item Mô hình chia sẻ học liệu nội bộ \textbf{StudyShare}, được xây dựng dành riêng cho sinh viên Trường Đại học Bách Khoa.
    \item Nền tảng cho phép người dùng đăng tải, tìm kiếm và trao đổi học liệu, sách và dụng cụ học tập trong phạm vi nội bộ, có xác thực tài khoản sinh viên nhằm đảm bảo tính an toàn và tin cậy.
\end{itemize}
&
\begin{itemize}
    \item Giảm thiểu thời gian tìm kiếm và tiếp cận học liệu cần thiết.
    \item Tăng tỷ lệ tái sử dụng tài nguyên học tập trong cộng đồng sinh viên.
    \item Giảm chi phí học tập trung bình của sinh viên thông qua việc tái sử dụng.
    \item Nâng cao tính an toàn, minh bạch và hiệu quả trong hoạt động trao đổi học liệu nội bộ.
\end{itemize}
\\
\hline
\end{tabular}
\end{table}

\subsubsection{Bán hàng (SELL)}

\begin{table}[H]
\centering
\renewcommand{\arraystretch}{1.3}
\begin{tabular}{|p{6cm}|p{4.5cm}|p{5cm}|}
\hline
\textbf{Lợi thế cạnh tranh (Advantage)} &
\textbf{Đối tượng người dùng (Customer/User)} &
\textbf{Chỉ số đo lường thành công (Metrics)} \\
\hline
\begin{itemize}
    \item Cộng đồng nội bộ được xác thực bằng email sinh viên (\texttt{@hcmut.edu.vn}).
    \item Bộ lọc đặc thù cho trao đổi giữa sinh viên, cho phép tra cứu và lọc theo môn học và ngành học.
    \item Môi trường đáng tin cậy có kiểm duyệt nhằm hạn chế bài đăng spam hoặc người ngoài trường.
    \item Tích hợp tính năng đánh giá và bình luận để tăng độ tin cậy trong quá trình trao đổi.
    \item Cho phép người dùng nhắn tin trực tiếp trong ứng dụng.
\end{itemize}
&
\begin{itemize}
    \item Sinh viên Đại học Bách Khoa TP.HCM từ năm nhất đến năm cuối và sinh viên sắp ra trường.
\end{itemize}
&
\begin{itemize}
    \item Số lượng học liệu hoặc dụng cụ được đăng tải mỗi tháng.
    \item Số lượng học liệu hoặc tài liệu trao đổi thành công mỗi tháng.
    \item Mức độ hài lòng của người dùng sau khi sử dụng.
    \item Tỷ lệ tái sử dụng học liệu qua các học kỳ.
    \item Tỷ lệ người dùng quay lại (Retention Rate) hàng tháng.
\end{itemize}
\\
\hline
\end{tabular}
\end{table}

\subsubsection{Mô hình kinh doanh (BUSINESS MODEL)}

\begin{table}[H]
\centering
\renewcommand{\arraystretch}{1.3}
\begin{tabular}{|p{5cm}|p{5cm}|p{5cm}|}
\hline
\textbf{Kênh tiếp cận (Channels)} &
\textbf{Chi phí vận hành (Costs)} &
\textbf{Doanh thu (Revenue)} \\
\hline
\begin{itemize}
    \item Truyền thông qua các \textbf{Câu lạc bộ sinh viên, fanpage Khoa, group Facebook chính thức của trường}.
    \item Marketing nội bộ trong khuôn viên trường (poster, sự kiện, truyền miệng).
    \item Tận dụng email sinh viên để gửi mail mời đăng ký tài khoản.
    \item Lập các kênh mạng xã hội để quảng bá (Tiktok, Facebook).
\end{itemize}
&
\begin{itemize}
    \item Chi phí hạ tầng hệ thống (hosting, lưu trữ dữ liệu, bảo mật).
    \item Chi phí quản lý cộng đồng và kiểm duyệt nội dung.
    \item Chi phí truyền thông nội bộ và hỗ trợ kỹ thuật.
\end{itemize}
&
\begin{itemize}
    \item Giai đoạn đầu (MVP): miễn phí hoàn toàn.
    \item Giai đoạn sau có thể mở rộng:
    \begin{itemize}
        \item Quảng cáo nội bộ (ưu tiên bài đăng).
        \item Gói ``đẩy tin'' hoặc ``nổi bật'' cho người bán.
        \item Phí dịch vụ nhỏ cho các giao dịch có hỗ trợ trung gian.
    \end{itemize}
\end{itemize}
\\
\hline
\end{tabular}
\end{table}

\subsubsection{Giá trị nổi bật so với các sản phẩm hiện có (Unique Value Proposition)}

\begin{table}[H]
\centering
\renewcommand{\arraystretch}{1.3}
\begin{tabular}{|p{6cm}|p{8cm}|}
\hline
\textbf{StudyShare} &
\textbf{Khác biệt so với nền tảng hiện tại} \\
\hline
Nội bộ sinh viên Bách Khoa (xác thực email \texttt{@hcmut.edu.vn}) &
So với Facebook, Chợ Tốt, Oreka -- nền tảng công khai, thiếu kiểm duyệt. \\
\hline
Tối ưu cho giao dịch học liệu và dụng cụ học tập &
Không bị nhiễu bởi các bài đăng ngoài mục đích học tập. \\
\hline
Chat trực tiếp trong ứng dụng &
Không cần qua nền tảng thứ ba, tiết kiệm thời gian. \\
\hline
Tập trung vào trải nghiệm di động đơn giản, thân thiện &
Phù hợp với thói quen sử dụng điện thoại của sinh viên. \\
\hline
Phát triển theo hướng cộng đồng học tập &
Hướng đến xây dựng văn hóa chia sẻ và hỗ trợ lẫn nhau giữa các khóa. \\
\hline
\end{tabular}
\end{table}
