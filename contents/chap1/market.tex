\subsection{Nghiên cứu thị trường}
Trong bối cảnh số lượng sinh viên Trường Đại học Bách Khoa ngày càng gia tăng, các sinh viên sau khi học các môn học như \textbf{thí nghiệm} hoặc các môn \textbf{chính trị}, \ldots{} có nhu cầu cao về \textbf{chia sẻ học liệu} và \textbf{trao đổi dụng cụ học tập}. Tuy nhiên, việc tìm kiếm và chia sẻ các học liệu hiện có vẫn gặp nhiều rào cản.

Phần lớn tài liệu được chia sẻ rải rác qua các bài đăng trên \textbf{Facebook}, \textbf{Chợ Tốt}, \textbf{Oreka}, \ldots{} Điều này dẫn đến tình trạng \textbf{phân tán thông tin}, \textbf{thiếu hệ thống} và \textbf{khó tiếp cận}. Sinh viên thường phải mất nhiều thời gian để lục tìm các tài liệu từ các khóa trước, trong khi nhu cầu trao đổi sách, giáo trình và dụng cụ học tập cũng không có một kênh tập trung rõ ràng.

Để phát triển một giải pháp phù hợp, nhóm tiến hành \textbf{nghiên cứu thị trường} nhằm:
\begin{itemize}
    \item Xác định nhu cầu thực tế và khó khăn của sinh viên, thông qua khảo sát các sinh viên tại Trường Đại học Bách Khoa.
    \item Phân tích các nền tảng tương tự hiện có trong nước, từ đó rút ra điểm mạnh, điểm yếu và các khoảng trống chưa được đáp ứng.
    \item Tìm kiếm cơ hội thị trường rõ ràng để \textbf{StudyShare} trở thành nền tảng chuyên biệt, tập trung vào việc hỗ trợ sinh viên Bách Khoa học tập hiệu quả hơn.
\end{itemize}

\subsubsection{Xây dựng bộ tiêu chí đánh giá}
Dựa trên \textbf{insight từ khảo sát sinh viên} về những khó khăn khi tìm kiếm, chia sẻ tài liệu và trao đổi dụng cụ học tập, nhóm xây dựng một \textbf{bộ tiêu chí đánh giá cụ thể} để phân tích 5 nền tảng cạnh tranh hiện tại như \textbf{Facebook Group}, \textbf{Facebook Marketplace}, \textbf{Telegram}, \textbf{ChoTot}, \textbf{Zalo}.

Bộ tiêu chí được chia thành \textbf{hai nhóm chính}:
\begin{itemize}
    \item \textbf{Nhóm Tính năng (Functional)}
    \item \textbf{Nhóm Trải nghiệm người dùng (UX)}
\end{itemize}

\begin{table}[H]
\centering
\renewcommand{\arraystretch}{1.2}
\begin{tabular}{|p{2.5cm}|p{4cm}|p{2cm}|p{6cm}|}
\hline
\textbf{Nhóm} & \textbf{Tiêu chí} & \textbf{Thang đo} & \textbf{Mô tả đánh giá} \\
\hline
Functional & Đăng bài trao đổi & 1--5 & 1 = Thao tác phức tạp; 5 = Đăng dễ dàng, hiển thị tốt \\
\hline
Functional & Tìm kiếm và lọc & 1--5 & 1 = Tìm cơ bản; 5 = Bộ lọc nâng cao \\
\hline
Functional & Đánh giá \& bình luận & 1--5 & 1 = Không hỗ trợ; 5 = Có đánh giá rõ ràng \\
\hline
Functional & Trao đổi dụng cụ học tập & 1--5 & 1 = Không phân loại; 5 = Có khu vực riêng \\
\hline
Functional & Xác thực sinh viên & Có/Không & Hỗ trợ xác minh sinh viên nội bộ \\
\hline
UX & Mức độ dễ sử dụng & 1--5 & 1 = Khó dùng; 5 = Giao diện thân thiện \\
\hline
UX & Ẩn bài đã hoàn tất & 1--5 & 1 = Không hỗ trợ; 5 = Có đánh dấu hoàn tất \\
\hline
UX & Phù hợp sinh viên & 1--5 & 1 = Ít sinh viên dùng; 5 = Phù hợp cao \\
\hline
\end{tabular}
\end{table}





\subsubsection{Facebook Group}
Facebook Group là kênh trao đổi phổ biến nhất của sinh viên Bách Khoa hiện nay. Hầu như mọi hoạt động trao đổi học liệu, dụng cụ học tập đều diễn ra trên các nhóm như “Chợ BK”, “Ký túc xá Bách Khoa”, “Hội Sinh viên năm 3 – năm 4”,… Đây là nền tảng gần gũi, có lượng người dùng lớn và phản ánh rõ nhất thói quen thực tế của sinh viên.

Facebook là nền tảng quen thuộc với sinh viên, thao tác đăng bài và tương tác (bình luận, nhắn tin) rất thuận tiện. Tuy nhiên, các group này không tập trung chuyên biệt cho việc trao đổi học liệu, nên bài đăng bị pha tạp, gây khó khăn khi tìm kiếm tài liệu.
\begin{figure}[H]
    \centering
    \begin{minipage}{0.23\textwidth}
        \centering
        \includegraphics[width=\linewidth]{images/chap1/market/fb_group_overview_1.png}
    \end{minipage}
    \begin{minipage}{0.23\textwidth}
        \centering
        \includegraphics[width=\linewidth]{images/chap1/market/fb_group_overview_2.png}
    \end{minipage}
    \begin{minipage}{0.23\textwidth}
        \centering
        \includegraphics[width=\linewidth]{images/chap1/market/fb_group_overview_3.png}
    \end{minipage}
    \begin{minipage}{0.23\textwidth}
        \centering
        \includegraphics[width=\linewidth]{images/chap1/market/fb_group_overview_4.png}
    \end{minipage}
\end{figure}

Quy trình đăng bài rất đơn giản (chỉ 3 bước: vào nhóm, soạn bài, đăng), giúp người dùng dễ dàng chia sẻ. Tuy vậy, nhiều bài đăng cũ không được xóa sau khi hoàn tất trao đổi, khiến nhóm chứa nhiều thông tin lỗi thời và gây mất thời gian khi tìm kiếm.
\begin{figure}[H]
    \centering
    \begin{minipage}{0.23\textwidth}
        \centering
        \includegraphics[width=\linewidth]{images/chap1/market/fb_group_post_1.png}
    \end{minipage}
    \begin{minipage}{0.23\textwidth}
        \centering
        \includegraphics[width=\linewidth]{images/chap1/market/fb_group_post_2.png}
    \end{minipage}
    \begin{minipage}{0.23\textwidth}
        \centering
        \includegraphics[width=\linewidth]{images/chap1/market/fb_group_post_3.png}
    \end{minipage}
\end{figure}

Chức năng tìm kiếm trong nhóm chỉ hỗ trợ theo từ khóa và bộ lọc cơ bản, không có hệ thống phân loại theo khoa, môn hay loại học liệu. Điều này làm việc tìm loại tài liệu cụ thể trở nên kém hiệu quả.
\begin{figure}[H]
    \centering
    \begin{minipage}{0.23\textwidth}
        \centering
        \includegraphics[width=\linewidth]{images/chap1/market/fb_group_search_1.png}
    \end{minipage}
    \begin{minipage}{0.23\textwidth}
        \centering
        \includegraphics[width=\linewidth]{images/chap1/market/fb_group_search_2.png}
    \end{minipage}
\end{figure}





\subsubsection{Facebook Marketplace}
Marketplace là tính năng thương mại tích hợp sẵn trong Facebook, được nhiều sinh viên sử dụng để bán hoặc pass lại sách, áo blouse, đồ học tập, đặc biệt khi cần tiếp cận nhiều người nhanh chóng.

Facebook Marketplace có giao diện rõ ràng, trực quan, thao tác xem bài và liên hệ người đăng rất dễ dàng. Tuy nhiên, nền tảng này không được thiết kế riêng cho học liệu, nên các bài đăng về tài liệu/dụng cụ học tập bị lẫn với nhiều loại mặt hàng khác.
\begin{figure}[H]
    \centering
    \begin{minipage}{0.23\textwidth}
        \centering
        \includegraphics[width=\linewidth]{images/chap1/market/fb_marketplace_overview_1.png}
    \end{minipage}
    \begin{minipage}{0.23\textwidth}
        \centering
        \includegraphics[width=\linewidth]{images/chap1/market/fb_marketplace_overview_2.png}
    \end{minipage}
    \begin{minipage}{0.23\textwidth}
        \centering
        \includegraphics[width=\linewidth]{images/chap1/market/fb_marketplace_overview_3.png}
    \end{minipage}
    \begin{minipage}{0.23\textwidth}
        \centering
        \includegraphics[width=\linewidth]{images/chap1/market/fb_marketplace_overview_4.png}
    \end{minipage}
\end{figure}

Quy trình đăng bài tương đối đơn giản, gồm 2 bước (điền thông tin cơ bản và đăng tải hình ảnh). Tuy vậy, việc phân loại không có mục riêng cho tài liệu học tập nên người đăng thường chọn danh mục ‘Khác’, dẫn đến khó tìm chính xác khi người mua tra cứu.
\begin{figure}[H]
    \centering
    \begin{minipage}{0.23\textwidth}
        \centering
        \includegraphics[width=\linewidth]{images/chap1/market/fb_marketplace_post_1.png}
    \end{minipage}
    \begin{minipage}{0.23\textwidth}
        \centering
        \includegraphics[width=\linewidth]{images/chap1/market/fb_marketplace_post_2.png}
    \end{minipage}
    \begin{minipage}{0.23\textwidth}
        \centering
        \includegraphics[width=\linewidth]{images/chap1/market/fb_marketplace_post_3.png}
    \end{minipage}
\end{figure}

Tính năng tìm kiếm hỗ trợ lọc theo vị trí, giá và danh mục, nhưng lại không hỗ trợ phân loại theo khoa, môn hay học kỳ. Vì vậy, việc tìm kiếm học liệu cụ thể trên Marketplace khá mất thời gian và dễ bị lẫn với các bài đăng không liên quan.
\begin{figure}[H]
    \centering
    \begin{minipage}{0.23\textwidth}
        \centering
        \includegraphics[width=\linewidth]{images/chap1/market/fb_marketplace_search_1.png}
    \end{minipage}
    \begin{minipage}{0.23\textwidth}
        \centering
        \includegraphics[width=\linewidth]{images/chap1/market/fb_marketplace_search_2.png}
    \end{minipage}
\end{figure}





\subsubsection{Oreka}
Oreka là nền tảng Việt Nam được xây dựng với mục tiêu thúc đẩy trao đổi và tái sử dụng đồ cũ, hướng đến cộng đồng sinh viên và người trẻ. Ứng dụng này cho phép người dùng đăng bài trao đổi, cho tặng hoặc mua bán đồ dùng cá nhân một cách nhanh chóng. Do có định hướng gần với mô hình “trao đổi học liệu – dụng cụ học tập”, Oreka được nhóm chọn để quan sát khả năng áp dụng cho sinh viên Bách Khoa.

Oreka có giao diện hiện đại, màu sắc tươi sáng, dễ nhìn và dễ thao tác. Đây là nền tảng chuyên về trao đổi, mua bán đồ dùng cá nhân, nên có phần tương đồng với mục tiêu trao đổi của sinh viên. Tuy nhiên, nền tảng chưa có danh mục riêng cho tài liệu hoặc dụng cụ học tập, nên sinh viên khó tìm được đúng đối tượng hoặc nội dung phù hợp.
\begin{figure}[H]
    \centering
    \begin{minipage}{0.23\textwidth}
        \centering
        \includegraphics[width=\linewidth]{images/chap1/market/oreka_overview_1.png}
    \end{minipage}
    \begin{minipage}{0.23\textwidth}
        \centering
        \includegraphics[width=\linewidth]{images/chap1/market/oreka_overview_2.png}
    \end{minipage}
    \begin{minipage}{0.23\textwidth}
        \centering
        \includegraphics[width=\linewidth]{images/chap1/market/oreka_overview_3.png}
    \end{minipage}
\end{figure}

Quy trình đăng bài trên Oreka tương đối phức tạp và nhiều bước. Điều này có thể tốn nhiều thời gian hơn khi người bán đăng bài nhưng nó cũng sẽ giúp họ mô tả chi tiết hơn để người mua dễ hiểu hơn. Tuy nhiên, đối với mục đích trao đổi tài liệu/dụng cụ học tập cho sinh viên thì việc này khá dư thừa và tốn thời gian. Đồng thời, người dùng sẽ phải chịu phí trung gian khi bán hoặc mua trên ứng dụng.
\begin{figure}[H]
    \centering
    \begin{minipage}{0.23\textwidth}
        \centering
        \includegraphics[width=\linewidth]{images/chap1/market/oreka_post_1.png}
    \end{minipage}
    \begin{minipage}{0.23\textwidth}
        \centering
        \includegraphics[width=\linewidth]{images/chap1/market/oreka_post_2.png}
    \end{minipage}
    \begin{minipage}{0.23\textwidth}
        \centering
        \includegraphics[width=\linewidth]{images/chap1/market/oreka_post_3.png}
    \end{minipage}
\end{figure}

Oreka có thanh tìm kiếm hoạt động ổn định và hỗ trợ lọc theo khu vực, tình trạng vật phẩm. Tuy nhiên, không có hệ thống lọc nâng cao theo chủ đề học tập hay loại học liệu, nên người dùng vẫn phải xem thủ công từng bài đăng để tìm vật phẩm phù hợp.
\begin{figure}[H]
    \centering
    \begin{minipage}{0.23\textwidth}
        \centering
        \includegraphics[width=\linewidth]{images/chap1/market/oreka_search_1.png}
    \end{minipage}
    \begin{minipage}{0.23\textwidth}
        \centering
        \includegraphics[width=\linewidth]{images/chap1/market/oreka_search_2.png}
    \end{minipage}
\end{figure}





\subsubsection{RaoVatVN}
RaoVatVN là một trong những trang rao vặt trực tuyến lâu đời và phổ biến tại Việt Nam, tập hợp nhiều danh mục như đồ dùng cá nhân, việc làm, bất động sản, xe cộ, v.v. Nền tảng này được chọn để khảo sát vì có hệ thống đăng tin và tìm kiếm đa dạng, phản ánh rõ mô hình giao dịch truyền thống trên web, giúp nhóm đánh giá điểm mạnh và hạn chế khi áp dụng cho môi trường sinh viên.

RaoVatVN có giao diện truyền thống, bố cục nhiều chữ và mục hiển thị dày đặc. Trang web hoạt động ổn định nhưng chưa thân thiện với người dùng trẻ và hiển thị chưa tối ưu trên thiết bị di động. Ngoài ra, do hướng đến nhiều lĩnh vực khác nhau (việc làm, bất động sản, dịch vụ…), nên không tập trung vào mảng trao đổi học liệu cho sinh viên. Có tính năng bình luận và đánh giá sản phẩm
\begin{figure}[H]
    \centering
    \begin{minipage}{0.23\textwidth}
        \centering
        \includegraphics[width=\linewidth]{images/chap1/market/raovat_overview_1.png}
    \end{minipage}
    \begin{minipage}{0.23\textwidth}
        \centering
        \includegraphics[width=\linewidth]{images/chap1/market/raovat_overview_2.png}
    \end{minipage}
    \begin{minipage}{0.23\textwidth}
        \centering
        \includegraphics[width=\linewidth]{images/chap1/market/raovat_overview_3.png}
    \end{minipage}
    \begin{minipage}{0.23\textwidth}
        \centering
        \includegraphics[width=\linewidth]{images/chap1/market/raovat_overview_4.png}
    \end{minipage}
\end{figure}

Quy trình đăng bài khá phức tạp do phải nhập rất nhiều thông tin, không thể đăng video mà chỉ có thể liên kết một video mà người dùng có sẵn. Tuy nhiên, người dùng được lựa chọn thời gian tin sẽ xuất hiện (tối đa 180 ngày) và các thông tin chi tiết khác có thể hỗ trợ cho người tìm kiếm biết được chi tiết hơn.
\begin{figure}[H]
    \centering
    \begin{minipage}{0.23\textwidth}
        \centering
        \includegraphics[width=\linewidth]{images/chap1/market/raovat_post_1.png}
    \end{minipage}
    \begin{minipage}{0.23\textwidth}
        \centering
        \includegraphics[width=\linewidth]{images/chap1/market/raovat_post_2.png}
    \end{minipage}
    \begin{minipage}{0.23\textwidth}
        \centering
        \includegraphics[width=\linewidth]{images/chap1/market/raovat_post_3.png}
    \end{minipage}
    \begin{minipage}{0.23\textwidth}
        \centering
        \includegraphics[width=\linewidth]{images/chap1/market/raovat_post_4.png}
    \end{minipage}
\end{figure}

Công cụ tìm kiếm chỉ hỗ trợ theo từ khóa và vị trí, người dùng có thể lọc theo giá, danh mục, thời gian đã đăng,...
\begin{figure}[H]
    \centering
    \begin{minipage}{0.23\textwidth}
        \centering
        \includegraphics[width=\linewidth]{images/chap1/market/raovat_search_1.png}
    \end{minipage}
    \begin{minipage}{0.23\textwidth}
        \centering
        \includegraphics[width=\linewidth]{images/chap1/market/raovat_search_2.png}
    \end{minipage}
\end{figure}





\subsubsection{Chợ Tốt}
Là nền tảng mua bán đồ cũ lớn nhất Việt Nam, Chợ Tốt có nhiều sinh viên sử dụng để pass sách giáo trình hoặc dụng cụ học tập. Đây là một đối thủ gián tiếp tiềm năng.

Chợ Tốt có giao diện khá trực quan, tập trung vào việc mua bán. Người dùng dễ dàng xem thông tin, hình ảnh và liên hệ với người bán qua tin nhắn hoặc gọi điện trực tiếp. Tuy nhiên, nền tảng này không được xây dựng riêng cho sinh viên nên thiếu các tính năng đặc thù như lọc theo môn học, học kỳ hoặc khoa.
\begin{figure}[H]
    \centering
    \begin{minipage}{0.23\textwidth}
        \centering
        \includegraphics[width=\linewidth]{images/chap1/market/chotot_overview_1.png}
    \end{minipage}
    \begin{minipage}{0.23\textwidth}
        \centering
        \includegraphics[width=\linewidth]{images/chap1/market/chotot_overview_2.png}
    \end{minipage}
    \begin{minipage}{0.23\textwidth}
        \centering
        \includegraphics[width=\linewidth]{images/chap1/market/chotot_overview_3.png}
    \end{minipage}
    \begin{minipage}{0.23\textwidth}
        \centering
        \includegraphics[width=\linewidth]{images/chap1/market/chotot_overview_4.png}
    \end{minipage}
\end{figure}

Đăng bài khá nhanh, gồm 3 bước đơn giản (chọn danh mục, nhập thông tin và tải hình ảnh). Tuy nhiên, do không có mục riêng cho học liệu/dụng cụ học tập, người dùng thường đăng vào mục ‘Đồ dùng văn phòng, công nông nghiệp’, dẫn đến việc tìm kiếm có thể bị lẫn với các loại sản phẩm khác.
\begin{figure}[H]
    \centering
    \begin{minipage}{0.23\textwidth}
        \centering
        \includegraphics[width=\linewidth]{images/chap1/market/chotot_post_1.png}
    \end{minipage}
    \begin{minipage}{0.23\textwidth}
        \centering
        \includegraphics[width=\linewidth]{images/chap1/market/chotot_post_2.png}
    \end{minipage}
    \begin{minipage}{0.23\textwidth}
        \centering
        \includegraphics[width=\linewidth]{images/chap1/market/chotot_post_3.png}
    \end{minipage}
\end{figure}

Chức năng tìm kiếm mạnh với từ khóa, giá và khu vực, nhưng không hỗ trợ phân loại chuyên sâu. Người dùng phải lọc thủ công và đọc từng bài, gây tốn thời gian khi cần tìm đúng loại tài liệu hoặc dụng cụ học tập cụ thể.
\begin{figure}[H]
    \centering
    \begin{minipage}{0.23\textwidth}
        \centering
        \includegraphics[width=\linewidth]{images/chap1/market/chotot_search_1.png}
    \end{minipage}
    \begin{minipage}{0.23\textwidth}
        \centering
        \includegraphics[width=\linewidth]{images/chap1/market/chotot_search_2.png}
    \end{minipage}
\end{figure}





\subsubsection{Tổng kết}

\begin{table}[H]
\centering
\renewcommand{\arraystretch}{1.2}
\begin{tabular}{|l|c|c|c|c|c|}
\hline
\textbf{Tiêu chí} &
\textbf{Facebook Group} &
\textbf{Facebook Marketplace} &
\textbf{Chợ Tốt} &
\textbf{Oreka} &
\textbf{RaoVatVN} \\
\hline
Đăng bài trao đổi & 4 & 4 & 4 & 3 & 3 \\
\hline
Tìm kiếm và lọc & 2 & 4 & 3 & 4 & 4 \\
\hline
Phù hợp với sinh viên & 5 & 3 & 3 & 3 & 2 \\
\hline
Đánh giá \& bình luận & 3 & 1 & 4 & 4 & 3 \\
\hline
Giúp sinh viên trao đổi dụng cụ học tập & 3 & 2 & 3 & 2 & 1 \\
\hline
Xác thực sinh viên nội bộ & 0 & 0 & 0 & 0 & 0 \\
\hline
Mức độ dễ sử dụng & 5 & 4 & 4 & 4 & 3 \\
\hline
Mức độ phổ biến / lan tỏa & 5 & 3 & 3 & 3 & 3 \\
\hline
Hỗ trợ ẩn bài đã hoàn tất & 2 & 4 & 4 & 4 & 3 \\
\hline
\end{tabular}
\end{table}





Qua quá trình khảo sát, nhóm nhận thấy các nền tảng hiện nay (Facebook Group, Marketplace, Chợ Tốt, Oreka, RaoVatVN) đều có điểm mạnh nhất định trong việc hỗ trợ người dùng đăng và tìm kiếm bài viết trao đổi tài liệu hoặc dụng cụ học tập:
\begin{itemize}
    \item Hầu hết sinh viên đã tương đối quen thuộc với Facebook, Chợ Tốt hoặc các nền tảng mua bán tương tự, nên việc thao tác đăng bài, xem bài và nhắn tin rất thuận tiện.
    \item Facebook Group là kênh có mức độ lan tỏa cao nhất nhờ lượng thành viên lớn, giúp bài đăng nhanh chóng tiếp cận cộng đồng sinh viên.
    \item Các nền tảng như Chợ Tốt, Marketplace hay Oreka có quy trình đăng bài rõ ràng, hỗ trợ hình ảnh và mô tả chi tiết, giúp người dùng dễ chia sẻ và tìm hiểu thông tin.
    \item Các ứng dụng như Marketplace, Chợ Tốt hay RaoVatVN cho phép lọc theo giá, tình trạng, vị trí -- hỗ trợ tìm kiếm tốt hơn so với Facebook Group.
\end{itemize}
Tổng thể, các nền tảng hiện nay phù hợp cho việc mua bán nhanh hoặc trao đổi vật dụng phổ thông, nhưng chưa thật sự đáp ứng được nhu cầu đặc thù của sinh viên trong việc chia sẻ học liệu và dụng cụ học tập.

Mặc dù có nhiều ưu điểm, các giải pháp hiện tại vẫn còn tồn tại những khoảng trống lớn khi xét đến nhu cầu của sinh viên đại học, đặc biệt trong môi trường học thuật:
\begin{itemize}
    \item Không có danh mục riêng cho học liệu / dụng cụ học tập: Người dùng phải đăng bài trong các danh mục chung như ``Đồ dùng văn phòng'' hoặc ``Khác'', khiến bài đăng dễ bị lẫn và khó tìm.
    \item Thiếu hệ thống lọc theo nhu cầu học tập: Không có cách nào lọc theo môn học, khoa, học kỳ hoặc chương trình đào tạo, làm việc tìm kiếm tài liệu cụ thể trở nên mất thời gian và kém hiệu quả.
    \item Không xác thực đối tượng sinh viên: Mọi người đều có thể đăng và xem bài, dẫn đến rủi ro về spam, lừa đảo hoặc đăng sai đối tượng. Điều này đặc biệt bất tiện khi sinh viên chỉ muốn trao đổi trong phạm vi trường hoặc khoa.
    \item Không có trạng thái quản lý bài đăng sau khi trao đổi: Nhiều bài cũ vẫn còn hiển thị dù giao dịch đã hoàn tất, gây nhiễu thông tin và giảm trải nghiệm người dùng.
\end{itemize}
Từ đó có thể thấy, nhu cầu trao đổi -- chia sẻ học liệu giữa sinh viên vẫn chưa được phục vụ đầy đủ bởi các nền tảng hiện có.

Dựa trên những khoảng trống nêu trên, nhóm nhận thấy cơ hội rõ ràng cho việc phát triển một ứng dụng chuyên biệt cho sinh viên trao đổi học liệu và dụng cụ học tập, với định hướng như sau:
\begin{itemize}
    \item \textbf{Tập trung vào nhu cầu học tập:} Ứng dụng chỉ hướng tới trao đổi, chia sẻ tài liệu dụng cụ học tập.
    \item \textbf{Cá nhân hóa theo môi trường học:} Có thể phân loại và lọc bài đăng theo khoa, môn học, học kỳ, giúp sinh viên nhanh chóng tìm đúng tài liệu cần thiết.
    \item \textbf{Xác thực người dùng theo trường / email sinh viên:} Chỉ cho phép đăng và tương tác trong phạm vi cộng đồng sinh viên, đảm bảo an toàn và đúng đối tượng.
    \item \textbf{Quản lý bài đăng thông minh:} Cung cấp trạng thái ``Đang trao đổi'' -- ``Đã hoàn tất'', hoặc tự động ẩn bài cũ để đảm bảo thông tin luôn cập nhật.
    \item \textbf{Tăng tính tương tác:} Thêm tính năng bình luận, hỏi đáp, đánh giá người đăng.
\end{itemize}
Các nền tảng hiện tại tuy đã đáp ứng một phần nhu cầu trao đổi của sinh viên, nhưng chưa có giải pháp nào thật sự tập trung vào khía cạnh học tập. Vì vậy, nhóm đề xuất phát triển một ứng dụng hướng tới cộng đồng sinh viên, giúp đăng -- tìm -- trao đổi học liệu và dụng cụ học tập nhanh chóng, đúng người, đúng nhu cầu.




































