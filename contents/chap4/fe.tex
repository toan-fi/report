\subsection{Frontend}
Phần Frontend của dự án sử dụng các công nghệ sau:
\begin{itemize}
    \item \textbf{React Native}
    \item \textbf{Expo}
    \item \textbf{TypeScript}
    \item \textbf{Node.js}
    \item \textbf{Expo Application Services (EAS)}
    \item \textbf{NativeWind (Tailwind CSS for React Native)}
    \item \textbf{React Context}
    \item \textbf{React Hooks}
    \item \textbf{Expo Router}
    \item \textbf{Axios}
    \item \textbf{RESTful API}
\end{itemize}
Frontend của ứng dụng được phát triển bằng \textbf{React Native kết hợp với Expo}, cho phép xây dựng ứng dụng di động đa nền tảng (Android và iOS) từ một codebase duy nhất. Việc sử dụng Expo giúp rút ngắn thời gian phát triển, đơn giản hóa quá trình cấu hình native và giảm chi phí thiết lập môi trường, từ đó nhóm có thể tập trung nhiều hơn vào logic nghiệp vụ và trải nghiệm người dùng.

Ứng dụng được viết bằng \textbf{TypeScript} trên nền \textbf{Node.js}, giúp tăng độ an toàn kiểu dữ liệu, phát hiện lỗi sớm trong quá trình phát triển và giảm thiểu lỗi runtime. TypeScript cũng hỗ trợ refactor hiệu quả và cải thiện khả năng bảo trì mã nguồn trong dài hạn.

\textbf{React Native} đóng vai trò là framework chính, mang lại hiệu năng gần với ứng dụng native và khả năng tái sử dụng code cao. \textbf{Expo} cung cấp sẵn các API truy cập thiết bị như camera, lưu trữ, quản lý quyền và networking mà không cần cấu hình native phức tạp, giúp quá trình phát triển diễn ra nhanh chóng và ổn định.

Quá trình build và phân phối ứng dụng được thực hiện thông qua \textbf{Expo Application Services (EAS)}. EAS cho phép build file APK trên nền tảng cloud, không phụ thuộc vào môi trường máy local, đồng thời hỗ trợ quản lý cấu hình theo từng môi trường (development, staging, production).

Về giao diện người dùng, ứng dụng sử dụng \textbf{NativeWind}, một framework styling dựa trên Tailwind CSS dành cho React Native. Cách tiếp cận utility-first giúp xây dựng giao diện nhanh, nhất quán, dễ tái sử dụng và thuận tiện cho việc mở rộng hệ thống UI.

Về kiến trúc, frontend được tổ chức theo \textbf{component-based architecture}, trong đó các màn hình, component tái sử dụng và logic nghiệp vụ được tách biệt rõ ràng. \textbf{React Context} được sử dụng để quản lý trạng thái toàn cục như thông tin người dùng và trạng thái đăng nhập, trong khi \textbf{React Hooks} xử lý trạng thái cục bộ của giao diện. Việc điều hướng trong ứng dụng được triển khai bằng \textbf{Expo Router}, áp dụng mô hình routing theo cấu trúc thư mục, giúp mã nguồn rõ ràng và dễ mở rộng.

Frontend giao tiếp với backend thông qua \textbf{RESTful API}, sử dụng \textbf{Axios} làm HTTP client. Axios được cấu hình tập trung trong một service layer để thống nhất cách gửi request, xử lý response và quản lý lỗi. Các thông tin cấu hình như API base URL được quản lý bằng biến môi trường của Expo, đảm bảo tính bảo mật và khả năng tách biệt giữa các môi trường triển khai.
